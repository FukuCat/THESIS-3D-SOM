\section{Quantitative Analysis}

Table 5.1 below shows the frequency count of each cluster per symphony. Each time a point is found in a cluster, the frequency count of the corresponding label is incremented. Table 5.5 shows the percentage of the same frequency distribution table of clusters per symphony. Cluster label A has the highest number of frequency with a total of 86983, followed by B with 65867, E with 65355, D with 60726, and lastly C with 51845. These cluster counts can be used to help verify the similarity of symphonies by, for example, checking the results from the above comparisons and seeing whether their cluster counts are truly similar. Since cluster counts do not incorporate the presence of time, unlike the previous comparisons, it is highly unlikely that comparisons made would be similar.and tabulated.

\begin{longtable}{|l|l|l|l|l|l|l|}
\caption{Frequency Cluster Count}
\label{my-label}\\
\hline
Row Labels & A & B & C & D & E & Grand Total \\ \hline
\endfirsthead
\endhead
%
P1C1S1 & 383 & 43 & 6 & 152 & 1 & 585 \\ \hline
P1C1S2 & 116 & 1609 & 2 & 16 & 710 & 2453 \\ \hline
P1C1S3 & 4 & 3 &  & 1 & 795 & 803 \\ \hline
P1C1S4 & 2 & 243 &  &  &  & 245 \\ \hline
P1C1S5 & 1 & 352 &  & 3 & 2 & 358 \\ \hline
P1C2S1 & 223 & 14 &  &  & 153 & 390 \\ \hline
P1C2S2 & 4 & 841 & 3 & 31 & 1 & 880 \\ \hline
P1C2S3 & 106 & 100 &  &  & 431 & 637 \\ \hline
P1C2S4 & 479 & 18 & 5 & 5 & 154 & 661 \\ \hline
P1C2S5 & 180 & 415 & 48 & 291 & 339 & 1273 \\ \hline
P1C3S1 & 1 & 41 &  &  & 596 & 638 \\ \hline
P1C3S2 & 450 & 73 & 5 & 32 & 71 & 631 \\ \hline
P1C3S3 &  & 90 &  & 1 & 227 & 318 \\ \hline
P1C3S4 & 2 & 10 &  & 387 &  & 399 \\ \hline
P1C3S5 & 276 & 2 &  & 1 & 23 & 302 \\ \hline
P1C4S1 & 58 & 99 & 20 & 82 & 22 & 281 \\ \hline
P1C4S2 & 83 & 109 &  & 2 & 783 & 977 \\ \hline
P1C4S3 & 297 & 286 & 83 & 408 & 57 & 1131 \\ \hline
P1C4S4 & 104 & 78 &  & 4 & 1344 & 1530 \\ \hline
P1C4S5 & 134 & 690 & 72 & 253 & 44 & 1193 \\ \hline
P1C5S1 & 33 & 952 & 2 & 1 & 24 & 1012 \\ \hline
P1C5S2 & 2 & 241 &  & 5 & 87 & 335 \\ \hline
P1C5S3 & 12 & 338 &  & 1 & 111 & 462 \\ \hline
P1C5S4 & 7 & 70 &  &  & 251 & 328 \\ \hline
P1C5S5 & 268 & 13 &  & 2 & 133 & 416 \\ \hline
P2C1S1 & 1369 & 643 & 228 & 414 & 121 & 2775 \\ \hline
P2C1S2 & 1055 & 827 & 77 & 398 & 180 & 2537 \\ \hline
P2C1S3 & 988 & 686 & 107 & 806 & 143 & 2730 \\ \hline
P2C1S4 & 1808 & 534 & 22 & 217 & 425 & 3006 \\ \hline
P2C1S5 & 478 & 148 & 18 & 90 & 232 & 966 \\ \hline
P2C2S1 & 86 & 91 & 40 & 40 & 191 & 448 \\ \hline
P2C2S2 & 180 & 258 & 60 & 270 & 110 & 878 \\ \hline
P2C2S3 & 336 & 1059 & 104 & 180 & 373 & 2052 \\ \hline
P2C2S4 & 353 & 287 & 39 & 32 & 454 & 1165 \\ \hline
P2C2S5 & 545 & 709 & 34 & 121 & 667 & 2076 \\ \hline
P2C3S1 & 279 & 369 & 12 & 22 & 249 & 931 \\ \hline
P2C3S2 & 72 & 206 & 2 & 8 & 804 & 1092 \\ \hline
P2C3S3 & 259 & 133 & 1 & 1 & 1218 & 1612 \\ \hline
P2C3S4 & 623 & 192 & 1 & 4 & 864 & 1684 \\ \hline
P2C3S5 & 504 & 123 & 6 & 81 & 599 & 1313 \\ \hline
P2C4S1 & 1396 & 330 & 16 & 44 & 1164 & 2950 \\ \hline
P2C4S2 & 764 & 612 & 421 & 265 & 845 & 2907 \\ \hline
P2C4S3 & 1082 & 293 & 71 & 162 & 913 & 2521 \\ \hline
P2C4S4 & 881 & 208 & 30 & 11 & 2112 & 3242 \\ \hline
P2C4S5 & 1091 & 546 & 492 & 271 & 771 & 3171 \\ \hline
P2C5S1 & 694 & 521 & 232 & 703 & 1446 & 3596 \\ \hline
P2C5S2 & 700 & 653 & 476 & 1144 & 80 & 3053 \\ \hline
P2C5S3 & 1375 & 373 & 539 & 792 & 281 & 3360 \\ \hline
P2C5S4 & 1145 & 1119 & 629 & 1408 & 26 & 4327 \\ \hline
P2C5S5 & 472 & 381 & 129 & 301 & 339 & 1622 \\ \hline
P3C1S1 & 514 & 411 & 87 & 296 & 46 & 1354 \\ \hline
P3C1S2 & 246 & 103 & 14 & 84 & 28 & 475 \\ \hline
P3C1S3 & 1460 & 258 & 65 & 112 & 498 & 2393 \\ \hline
P3C1S4 & 1032 & 102 & 15 & 29 & 933 & 2111 \\ \hline
P3C1S5 & 47 & 116 & 2 & 3 & 1109 & 1277 \\ \hline
P3C2S1 & 36 & 3 & 1653 & 237 & 2 & 1931 \\ \hline
P3C2S2 & 371 & 70 & 10 & 1 & 554 & 1006 \\ \hline
P3C2S3 & 461 & 41 & 9 &  & 531 & 1042 \\ \hline
P3C2S4 & 1048 & 102 & 80 & 199 & 574 & 2003 \\ \hline
P3C2S5 & 487 & 73 & 434 & 444 & 15 & 1453 \\ \hline
P3C3S1 & 32 &  & 1351 & 724 &  & 2107 \\ \hline
P3C3S2 & 18 &  & 1068 & 598 & 5 & 1689 \\ \hline
P3C3S3 & 173 &  & 1568 & 245 & 4 & 1990 \\ \hline
P3C3S4 & 54 & 10 & 3109 & 1114 & 4 & 4291 \\ \hline
P3C3S5 & 42 &  & 1139 & 521 &  & 1702 \\ \hline
P3C4S1 & 1028 & 139 & 53 & 378 & 462 & 2060 \\ \hline
P3C4S2 & 414 & 316 & 118 & 250 & 90 & 1188 \\ \hline
P3C4S3 & 2275 & 340 & 135 & 192 & 346 & 3288 \\ \hline
P3C4S4 & 1021 & 522 & 108 & 145 & 469 & 2265 \\ \hline
P3C4S5 & 602 & 194 & 6 & 13 & 879 & 1694 \\ \hline
P3C5S1 & 592 & 311 & 377 & 347 & 317 & 1944 \\ \hline
P3C5S2 & 1493 & 740 & 244 & 546 & 101 & 3124 \\ \hline
P3C5S3 & 1753 & 286 & 129 & 329 & 22 & 2519 \\ \hline
P3C5S4 & 713 & 126 & 14 & 28 & 938 & 1819 \\ \hline
P3C5S5 & 27 & 70 & 2 & 4 & 723 & 826 \\ \hline
P4C1S1 & 429 & 316 & 106 & 186 & 143 & 1180 \\ \hline
P4C1S2 & 820 & 2985 & 264 & 1919 & 58 & 6046 \\ \hline
P4C1S3 & 1145 & 1717 & 357 & 621 & 307 & 4147 \\ \hline
P4C1S4 & 1088 & 519 & 54 & 107 & 346 & 2114 \\ \hline
P4C1S5 & 612 & 215 & 2 & 88 & 424 & 1341 \\ \hline
P4C2S1 & 438 & 182 & 1 & 100 & 2682 & 3403 \\ \hline
P4C2S2 & 1456 & 575 & 207 & 490 & 115 & 2843 \\ \hline
P4C2S3 & 2016 & 1059 & 103 & 332 & 524 & 4034 \\ \hline
P4C2S4 & 1329 & 354 & 17 & 46 & 804 & 2550 \\ \hline
P4C2S5 & 593 & 94 &  & 15 & 848 & 1550 \\ \hline
P4C3S1 & 15 &  & 2838 & 1617 & 21 & 4491 \\ \hline
P4C3S2 & 7 &  & 2373 & 1664 & 8 & 4052 \\ \hline
P4C3S3 & 46 &  & 168 & 5139 &  & 5353 \\ \hline
P4C3S4 & 12 & 253 & 153 & 3404 & 12 & 3834 \\ \hline
P4C3S5 & 4 & 3885 &  & 62 & 99 & 4050 \\ \hline
P4C4S1 & 889 & 497 &  & 15 & 1337 & 2738 \\ \hline
P4C4S2 & 1195 & 566 & 2 & 201 & 910 & 2874 \\ \hline
P4C4S3 & 1875 & 66 & 212 & 287 & 342 & 2782 \\ \hline
P4C4S4 & 138 & 1 & 3393 & 25 &  & 3557 \\ \hline
P4C4S5 & 40 & 7 & 3044 & 60 &  & 3151 \\ \hline
P4C5S1 & 379 & 367 & 243 & 401 & 70 & 1460 \\ \hline
P4C5S2 & 673 & 250 & 204 & 327 & 140 & 1594 \\ \hline
P4C5S3 & 1866 & 331 & 157 & 398 & 408 & 3160 \\ \hline
P4C5S4 & 672 & 379 & 7 & 104 & 476 & 1638 \\ \hline
P4C5S5 & 168 & 354 & 33 & 87 & 842 & 1484 \\ \hline
P5C1S1 & 46 & 471 & 1397 & 3291 & 101 & 5306 \\ \hline
P5C1S2 & 158 & 15 & 3637 & 1511 &  & 5321 \\ \hline
P5C1S3 & 150 & 21 & 2394 & 1556 &  & 4121 \\ \hline
P5C1S4 & 472 & 11 & 3292 & 1729 &  & 5504 \\ \hline
P5C1S5 & 652 & 13 & 2826 & 2141 &  & 5632 \\ \hline
P5C2S1 & 2968 & 1071 & 756 & 822 & 75 & 5692 \\ \hline
P5C2S2 & 2594 & 683 & 1211 & 1497 & 157 & 6142 \\ \hline
P5C2S3 & 2612 & 217 & 63 & 201 & 1489 & 4582 \\ \hline
P5C2S4 & 341 & 1070 & 94 & 114 & 4240 & 5859 \\ \hline
P5C2S5 & 3811 & 854 & 690 & 526 & 385 & 3199 \\ \hline
P5C3S1 & 1194 & 1907 & 618 & 1215 & 493 & 5427 \\ \hline
P5C3S2 & 2590 & 1264 & 417 & 734 & 564 & 5569 \\ \hline
P5C3S3 & 2975 & 1009 & 212 & 786 & 419 & 5401 \\ \hline
P5C3S4 & 1460 & 534 & 20 & 730 & 1418 & 4162 \\ \hline
P5C3S5 & 2023 & 707 & 8 & 299 & 2572 & 5609 \\ \hline
P5C4S1 & 585 & 1076 & 911 & 1186 & 165 & 3923 \\ \hline
P5C4S2 & 292 & 1955 & 1144 & 1943 & 111 & 5445 \\ \hline
P5C4S3 & 624 & 2737 & 428 & 1329 & 552 & 5670 \\ \hline
P5C4S4 & 641 & 2713 & 719 & 1160 & 286 & 5519 \\ \hline
P5C4S5 & 1242 & 2513 & 299 & 561 & 424 & 5039 \\ \hline
P5C5S1 & 234 & 2447 & 6 & 360 & 3036 & 6083 \\ \hline
P5C5S2 & 58 & 3071 & 22 & 26 & 938 & 4115 \\ \hline
P5C5S3 & 2613 & 665 & 8 & 333 & 2554 & 6173 \\ \hline
P5C5S4 & 251 & 762 &  & 170 & 4067 & 5250 \\ \hline
P5C5S5 & 793 & 745 & 913 & 2509 & 777 & 5737 \\ \hline
Grand Total & 86983 & 65867 & 51845 & 60726 & 65355 & 330776 \\ \hline
\end{longtable}

\begin{longtable}{|l|l|l|l|l|l|}
\caption{Percentage Cluster Count}
\label{my-label}\\
\hline
Row Labels & A & B & C & D & E \\ \hline
\endfirsthead
\endhead
%
P1C1S1 & 65\% & 7\% & 1\% & 26\% & 0\% \\ \hline
P1C1S2 & 5\% & 66\% & 0\% & 1\% & 29\% \\ \hline
P1C1S3 & 0\% & 0\% & 0\% & 0\% & 99\% \\ \hline
P1C1S4 & 1\% & 99\% & 0\% & 0\% & 0\% \\ \hline
P1C1S5 & 0\% & 98\% & 0\% & 1\% & 1\% \\ \hline
P1C2S1 & 57\% & 4\% & 0\% & 0\% & 39\% \\ \hline
P1C2S2 & 0\% & 96\% & 0\% & 4\% & 0\% \\ \hline
P1C2S3 & 17\% & 16\% & 0\% & 0\% & 68\% \\ \hline
P1C2S4 & 72\% & 3\% & 1\% & 1\% & 23\% \\ \hline
P1C2S5 & 14\% & 33\% & 4\% & 23\% & 27\% \\ \hline
P1C3S1 & 0\% & 6\% & 0\% & 0\% & 93\% \\ \hline
P1C3S2 & 71\% & 12\% & 1\% & 5\% & 11\% \\ \hline
P1C3S3 & 0\% & 28\% & 0\% & 0\% & 71\% \\ \hline
P1C3S4 & 1\% & 3\% & 0\% & 97\% & 0\% \\ \hline
P1C3S5 & 91\% & 1\% & 0\% & 0\% & 8\% \\ \hline
P1C4S1 & 21\% & 35\% & 7\% & 29\% & 8\% \\ \hline
P1C4S2 & 8\% & 11\% & 0\% & 0\% & 80\% \\ \hline
P1C4S3 & 26\% & 25\% & 7\% & 36\% & 5\% \\ \hline
P1C4S4 & 7\% & 5\% & 0\% & 0\% & 88\% \\ \hline
P1C4S5 & 11\% & 58\% & 6\% & 21\% & 4\% \\ \hline
P1C5S1 & 3\% & 94\% & 0\% & 0\% & 2\% \\ \hline
P1C5S2 & 1\% & 72\% & 0\% & 1\% & 26\% \\ \hline
P1C5S3 & 3\% & 73\% & 0\% & 0\% & 24\% \\ \hline
P1C5S4 & 2\% & 21\% & 0\% & 0\% & 77\% \\ \hline
P1C5S5 & 64\% & 3\% & 0\% & 0\% & 32\% \\ \hline
P2C1S1 & 49\% & 23\% & 8\% & 15\% & 4\% \\ \hline
P2C1S2 & 42\% & 33\% & 3\% & 16\% & 7\% \\ \hline
P2C1S3 & 36\% & 25\% & 4\% & 30\% & 5\% \\ \hline
P2C1S4 & 60\% & 18\% & 1\% & 7\% & 14\% \\ \hline
P2C1S5 & 49\% & 15\% & 2\% & 9\% & 24\% \\ \hline
P2C2S1 & 19\% & 20\% & 9\% & 9\% & 43\% \\ \hline
P2C2S2 & 21\% & 29\% & 7\% & 31\% & 13\% \\ \hline
P2C2S3 & 16\% & 52\% & 5\% & 9\% & 18\% \\ \hline
P2C2S4 & 30\% & 25\% & 3\% & 3\% & 39\% \\ \hline
P2C2S5 & 26\% & 34\% & 2\% & 6\% & 32\% \\ \hline
P2C3S1 & 30\% & 40\% & 1\% & 2\% & 27\% \\ \hline
P2C3S2 & 7\% & 19\% & 0\% & 1\% & 74\% \\ \hline
P2C3S3 & 16\% & 8\% & 0\% & 0\% & 76\% \\ \hline
P2C3S4 & 37\% & 11\% & 0\% & 0\% & 51\% \\ \hline
P2C3S5 & 38\% & 9\% & 0\% & 6\% & 46\% \\ \hline
P2C4S1 & 47\% & 11\% & 1\% & 1\% & 39\% \\ \hline
P2C4S2 & 26\% & 21\% & 14\% & 9\% & 29\% \\ \hline
P2C4S3 & 43\% & 12\% & 3\% & 6\% & 36\% \\ \hline
P2C4S4 & 27\% & 6\% & 1\% & 0\% & 65\% \\ \hline
P2C4S5 & 34\% & 17\% & 16\% & 9\% & 24\% \\ \hline
P2C5S1 & 19\% & 14\% & 6\% & 20\% & 40\% \\ \hline
P2C5S2 & 23\% & 21\% & 16\% & 37\% & 3\% \\ \hline
P2C5S3 & 41\% & 11\% & 16\% & 24\% & 8\% \\ \hline
P2C5S4 & 26\% & 26\% & 15\% & 33\% & 1\% \\ \hline
P2C5S5 & 29\% & 23\% & 8\% & 19\% & 21\% \\ \hline
P3C1S1 & 38\% & 30\% & 6\% & 22\% & 3\% \\ \hline
P3C1S2 & 52\% & 22\% & 3\% & 18\% & 6\% \\ \hline
P3C1S3 & 61\% & 11\% & 3\% & 5\% & 21\% \\ \hline
P3C1S4 & 49\% & 5\% & 1\% & 1\% & 44\% \\ \hline
P3C1S5 & 4\% & 9\% & 0\% & 0\% & 87\% \\ \hline
P3C2S1 & 2\% & 0\% & 86\% & 12\% & 0\% \\ \hline
P3C2S2 & 37\% & 7\% & 1\% & 0\% & 55\% \\ \hline
P3C2S3 & 44\% & 4\% & 1\% & 0\% & 51\% \\ \hline
P3C2S4 & 52\% & 5\% & 4\% & 10\% & 29\% \\ \hline
P3C2S5 & 34\% & 5\% & 30\% & 31\% & 1\% \\ \hline
P3C3S1 & 2\% & 0\% & 64\% & 34\% & 0\% \\ \hline
P3C3S2 & 1\% & 0\% & 63\% & 35\% & 0\% \\ \hline
P3C3S3 & 9\% & 0\% & 79\% & 12\% & 0\% \\ \hline
P3C3S4 & 1\% & 0\% & 72\% & 26\% & 0\% \\ \hline
P3C3S5 & 2\% & 0\% & 67\% & 31\% & 0\% \\ \hline
P3C4S1 & 50\% & 7\% & 3\% & 18\% & 22\% \\ \hline
P3C4S2 & 35\% & 27\% & 10\% & 21\% & 8\% \\ \hline
P3C4S3 & 69\% & 10\% & 4\% & 6\% & 11\% \\ \hline
P3C4S4 & 45\% & 23\% & 5\% & 6\% & 21\% \\ \hline
P3C4S5 & 36\% & 11\% & 0\% & 1\% & 52\% \\ \hline
P3C5S1 & 30\% & 16\% & 19\% & 18\% & 16\% \\ \hline
P3C5S2 & 48\% & 24\% & 8\% & 17\% & 3\% \\ \hline
P3C5S3 & 70\% & 11\% & 5\% & 13\% & 1\% \\ \hline
P3C5S4 & 39\% & 7\% & 1\% & 2\% & 52\% \\ \hline
P3C5S5 & 3\% & 8\% & 0\% & 0\% & 88\% \\ \hline
P4C1S1 & 36\% & 27\% & 9\% & 16\% & 12\% \\ \hline
P4C1S2 & 14\% & 49\% & 4\% & 32\% & 1\% \\ \hline
P4C1S3 & 28\% & 41\% & 9\% & 15\% & 7\% \\ \hline
P4C1S4 & 51\% & 25\% & 3\% & 5\% & 16\% \\ \hline
P4C1S5 & 46\% & 16\% & 0\% & 7\% & 32\% \\ \hline
P4C2S1 & 13\% & 5\% & 0\% & 3\% & 79\% \\ \hline
P4C2S2 & 51\% & 20\% & 7\% & 17\% & 4\% \\ \hline
P4C2S3 & 50\% & 26\% & 3\% & 8\% & 13\% \\ \hline
P4C2S4 & 52\% & 14\% & 1\% & 2\% & 32\% \\ \hline
P4C2S5 & 38\% & 6\% & 0\% & 1\% & 55\% \\ \hline
P4C3S1 & 0\% & 0\% & 63\% & 36\% & 0\% \\ \hline
P4C3S2 & 0\% & 0\% & 59\% & 41\% & 0\% \\ \hline
P4C3S3 & 1\% & 0\% & 3\% & 96\% & 0\% \\ \hline
P4C3S4 & 0\% & 7\% & 4\% & 89\% & 0\% \\ \hline
P4C3S5 & 0\% & 96\% & 0\% & 2\% & 2\% \\ \hline
P4C4S1 & 32\% & 18\% & 0\% & 1\% & 49\% \\ \hline
P4C4S2 & 42\% & 20\% & 0\% & 7\% & 32\% \\ \hline
P4C4S3 & 67\% & 2\% & 8\% & 10\% & 12\% \\ \hline
P4C4S4 & 4\% & 0\% & 95\% & 1\% & 0\% \\ \hline
P4C4S5 & 1\% & 0\% & 97\% & 2\% & 0\% \\ \hline
P4C5S1 & 26\% & 25\% & 17\% & 27\% & 5\% \\ \hline
P4C5S2 & 42\% & 16\% & 13\% & 21\% & 9\% \\ \hline
P4C5S3 & 59\% & 10\% & 5\% & 13\% & 13\% \\ \hline
P4C5S4 & 41\% & 23\% & 0\% & 6\% & 29\% \\ \hline
P4C5S5 & 11\% & 24\% & 2\% & 6\% & 57\% \\ \hline
P5C1S1 & 1\% & 9\% & 26\% & 62\% & 2\% \\ \hline
P5C1S2 & 3\% & 0\% & 68\% & 28\% & 0\% \\ \hline
P5C1S3 & 4\% & 1\% & 58\% & 38\% & 0\% \\ \hline
P5C1S4 & 9\% & 0\% & 60\% & 31\% & 0\% \\ \hline
P5C1S5 & 12\% & 0\% & 50\% & 38\% & 0\% \\ \hline
P5C2S1 & 52\% & 19\% & 13\% & 14\% & 1\% \\ \hline
P5C2S2 & 42\% & 11\% & 20\% & 24\% & 3\% \\ \hline
P5C2S3 & 57\% & 5\% & 1\% & 4\% & 32\% \\ \hline
P5C2S4 & 6\% & 18\% & 2\% & 2\% & 72\% \\ \hline
P5C2S5 & 119\% & 27\% & 22\% & 16\% & 12\% \\ \hline
P5C3S1 & 22\% & 35\% & 11\% & 22\% & 9\% \\ \hline
P5C3S2 & 47\% & 23\% & 7\% & 13\% & 10\% \\ \hline
P5C3S3 & 55\% & 19\% & 4\% & 15\% & 8\% \\ \hline
P5C3S4 & 35\% & 13\% & 0\% & 18\% & 34\% \\ \hline
P5C3S5 & 36\% & 13\% & 0\% & 5\% & 46\% \\ \hline
P5C4S1 & 15\% & 27\% & 23\% & 30\% & 4\% \\ \hline
P5C4S2 & 5\% & 36\% & 21\% & 36\% & 2\% \\ \hline
P5C4S3 & 11\% & 48\% & 8\% & 23\% & 10\% \\ \hline
P5C4S4 & 12\% & 49\% & 13\% & 21\% & 5\% \\ \hline
P5C4S5 & 25\% & 50\% & 6\% & 11\% & 8\% \\ \hline
P5C5S1 & 4\% & 40\% & 0\% & 6\% & 50\% \\ \hline
P5C5S2 & 1\% & 75\% & 1\% & 1\% & 23\% \\ \hline
P5C5S3 & 42\% & 11\% & 0\% & 5\% & 41\% \\ \hline
P5C5S4 & 5\% & 15\% & 0\% & 3\% & 77\% \\ \hline
P5C5S5 & 14\% & 13\% & 16\% & 44\% & 14\% \\ \hline
\end{longtable}

Listed below in table 5.3 are the results of using Manhattan distance for pairwise comparison of symphonies with respect to the time dimension of each symphonies. Each point in time of the t-SNE result for a symphony is matched with a point of another symphony and the distance between these two points is computed. The process is simply repeated until the end of either symphony is reached. Truncation is thus performed on the longer string to match the lengths of the two symphonies being compared. The resulting collection of distances are thus summed 

The results show that Bach’s sinfonia no. 7 and sinfonia no. 4 are the most similar among the entire list of symphonies, followed by Gabrieli’s canzon noni toni a 12-correggio with Viadana’s sinfonia la bolognese, and so on. Contrary to expectation that symphonies from the same composer would be more similar compared to those of other composers, a lot of entries in the top 30 have symphonies that come from different composers. Interestingly, a lot of entries in the top 30 also come from Bach’s sinfonia no. 7, which could very well indicate that its points may be well scattered throughout the map when plotted or the points could be positioned near the center of the map. Verifying this, we look at Bach’s graph in Figure 5.6 and the green points, which represent Bach’s sinfonia no. 7 are somewhat located near the middle of the graph. They are not, however, scattered throughout the map. Viadana’s Sinfonia la Bergamasca is the second most number of matches in the top 30 and based on its graph in Figure 5.10, it can be seen that this symphony is somewhat scattered but the main bulk of its points are still near the middle. Seeing this result may well indicate that symphonies located near the middle when graphed may have a stronger tendency to have higher matches with more symphonies when comparing through distances of points via time. To furr verify this method of comparison between symphonies, we shall compare these results with the results of other metrics later on. The results of this metric together with other succeeding partial results below can be further viewed in their respective spreadsheet files as the results of these pairwise comparisons have a total number of 125x125 rows.

\begin{longtable}{|l|l|l|}
\caption{Manhattan Distance Top Results: Full Length Summed}
\label{my-label}\\
\hline
Symphony A & Symphony B & Distance Sum \\ \hline
\endfirsthead
\endhead
%
P1C1S4 & P1C1S5 & 269.9242 \\ \hline
P1C3S3 & P1C5S4 & 598.8077 \\ \hline
P1C1S4 & P1C5S3 & 776.0105 \\ \hline
P1C1S4 & P1C3S3 & 787.6107 \\ \hline
P1C1S4 & P1C5S4 & 807.6517 \\ \hline
P1C4S1 & P1C4S3 & 947.415 \\ \hline
P1C3S3 & P1C5S3 & 967.2533 \\ \hline
P1C1S5 & P1C5S3 & 970.1439 \\ \hline
P1C4S1 & P1C4S5 & 993.4798 \\ \hline
P1C1S4 & P5C5S2 & 999.3968 \\ \hline
P1C4S1 & P4C1S2A & 1010.578 \\ \hline
P1C1S4 & P1C2S2 & 1017.719 \\ \hline
P1C1S4 & P1C5S2 & 1027.048 \\ \hline
P1C4S1 & P5C3S3A & 1034.093 \\ \hline
P1C4S1 & P2C5S2A & 1045.884 \\ \hline
P1C4S1 & P2C2S2A & 1059.302 \\ \hline
P1C4S1 & P2C5S4A & 1074.551 \\ \hline
P1C4S1 & P4C1S3A & 1104.918 \\ \hline
P1C4S1 & P3C5S3 & 1117.526 \\ \hline
P1C4S1 & P4C2S3A & 1130.082 \\ \hline
P1C4S1 & P3C1S1 & 1153.61 \\ \hline
P1C4S1 & P2C1S3 & 1156.801 \\ \hline
P1C4S1 & P3C1S2 & 1158.137 \\ \hline
P1C1S4 & P1C2S5 & 1167.234 \\ \hline
P1C4S1 & P4C1S4 & 1181.949 \\ \hline
P1C4S1 & P2C1S2 & 1186.405 \\ \hline
P1C4S1 & P3C5S2 & 1198.531 \\ \hline
P1C1S4 & P1C3S1 & 1222.162 \\ \hline
P1C4S1 & P2C3S1A & 1225.814 \\ \hline
P1C4S1 & P2C1S1 & 1234.873 \\ \hline
P1C4S1 & P4C1S1A & 1238.681 \\ \hline
\end{longtable}

Table 5.4 below shows the Manhattan distance comparison of symphonies, but this time, through the transitions only. Interestingly, a lot of the rankings of the results matched that of the earlier comparisons like for example, a lot of the top 30 results here still showed Bach’s sinfonia no. 7.

\begin{longtable}{|l|l|l|}
\caption{Manhattan Distance Result: Transitions Summed}
\label{my-label}\\
\hline
Symphony A & Symphony B & Distance Sum \\ \hline
\endfirsthead
%
\endhead
%
P2C1S3 & P1C1S4 & 6.342487 \\ \hline
P2C5S3A & P1C1S4 & 7.09482 \\ \hline
P3C1S3 & P1C1S4 & 7.191891 \\ \hline
P2C4S2A & P1C1S4 & 7.276127 \\ \hline
P1C1S4 & P1C1S5 & 7.354168 \\ \hline
P4C2S1A & P1C1S4 & 7.489208 \\ \hline
P1C2S5 & P1C1S4 & 7.49975 \\ \hline
P2C3S1A & P1C1S4 & 7.522169 \\ \hline
P1C1S5 & P1C1S4 & 7.667543 \\ \hline
P1C1S4 & P1C5S5 & 7.682119 \\ \hline
P1C1S4 & P2C3S2A & 7.769781 \\ \hline
P2C3S2A & P1C1S4 & 7.805628 \\ \hline
P2C1S4 & P1C1S4 & 7.806015 \\ \hline
P2C2S5A & P1C1S4 & 7.961077 \\ \hline
P2C3S3A & P1C1S4 & 7.981119 \\ \hline
P5C2S3 & P1C1S4 & 8.046022 \\ \hline
P1C2S3 & P1C2S3 & 8.079905 \\ \hline
P4C1S1A & P1C1S4 & 8.13656 \\ \hline
P4C1S5A & P1C1S4 & 8.149289 \\ \hline
P2C4S1A & P1C1S4 & 8.167969 \\ \hline
P2C2S1 & P1C1S4 & 8.23078 \\ \hline
P4C5S4 & P1C1S4 & 8.290154 \\ \hline
P2C5S1A & P1C1S4 & 8.318904 \\ \hline
P5C5S4 & P1C1S4 & 8.383004 \\ \hline
P1C1S4 & P2C5S1A & 8.563907 \\ \hline
P1C1S4 & P1C1S2 & 8.621842 \\ \hline
P1C1S4 & P5C3S1A & 8.723814 \\ \hline
P1C4S5 & P1C1S4 & 8.732073 \\ \hline
P4C4S2 & P1C1S4 & 8.783347 \\ \hline
P1C1S4 & P3C1S5 & 8.82796 \\ \hline
\end{longtable}

If we take the the top results from the Manhattan distance above, with Bach’s sinfonia no. 7 (P1C1S4) and sinfonia no. 4 (P1C1S5) being the top for Manhattan distance comparison, we can see below in Table 5.4 that indeed their cluster couts are somewhat similar with the majority of the points being in cluster B. Sinfonia no. 7 has 99\% of points in B while sinfonia no. 4 has 98\% of points in B.  The second closest with the Manhattan result, Gabrieli’s canzon noni toni a 12-correggio (P1C3S3) and Viadana’s sinfonia la bolognese (P1C5S4), again show the same result with their cluster counts being very similar in that both of them had near 75\% of points in cluster E and nea 25\% in cluster B.

Bach’s sinfonia no. 7 (P1C1S4) which produced a lot of high matches with other symphonies is also coincidentally the shortest symphony among the dataset and this may indicate that truncation indeed is not a reliable method of matching the lengths of the symphonies for comparison or it could also mean that sinfonia no. 7 simply has a lot in common with most symphonies. In order to further verify this, we repeat the Manhattan distance comparison, but this time, by taking the average of the distances of all the points that were compared instead of simply adding them all up. Table 5.5 below shows the top 30 results. Bach’s sinfonia no. 7 can still be seen in some entries but the entries this time around have more variance in them unlike the total distance from before.

\begin{longtable}{|l|l|l|}
\caption{Manhattan Distance Result: Distances Mean}
\label{my-label}\\
\hline
Symphony A & Symphony B & Distance Mean \\ \hline
\endfirsthead
%
\endhead
%
P1C1S4 & P1C1S5 & 1.106247 \\ \hline
P1C3S3 & P1C5S4 & 1.888983 \\ \hline
P3C3S2 & P3C3S5 & 2.709862 \\ \hline
P1C1S5 & P1C5S3 & 2.71749 \\ \hline
P4C4S4 & P4C4S5 & 2.783062 \\ \hline
P3C2S1 & P3C3S1 & 2.885411 \\ \hline
P3C3S1 & P3C3S2 & 2.894063 \\ \hline
P1C3S3 & P1C5S3 & 3.051272 \\ \hline
P3C2S1 & P3C3S4 & 3.158728 \\ \hline
P1C1S4 & P1C5S3 & 3.180371 \\ \hline
P3C3S1 & P3C3S3 & 3.197033 \\ \hline
P3C3S2 & P3C3S3 & 3.213563 \\ \hline
P1C1S4 & P1C3S3 & 3.227913 \\ \hline
P3C3S1 & P3C3S4 & 3.303385 \\ \hline
P1C1S4 & P1C5S4 & 3.310048 \\ \hline
P1C1S1 & P3C5S3 & 3.356575 \\ \hline
P1C4S1 & P1C4S3 & 3.383625 \\ \hline
P3C2S1 & P3C3S2 & 3.384755 \\ \hline
P3C1S4 & P4C2S1A & 3.422252 \\ \hline
P1C1S1 & P1C3S2 & 3.433787 \\ \hline
P2C4S4A & P4C2S1A & 3.438686 \\ \hline
P1C1S1 & P3C1S1 & 3.472338 \\ \hline
P3C1S4 & P3C5S4 & 3.541771 \\ \hline
P3C4S5 & P4C2S1A & 3.548003 \\ \hline
P1C4S1 & P1C4S5 & 3.548142 \\ \hline
P1C5S1 & P4C3S5 & 3.583475 \\ \hline
P3C5S4 & P4C2S1A & 3.58519 \\ \hline
P3C2S1 & P3C3S3 & 3.594948 \\ \hline
P1C4S1 & P4C1S2A & 3.609206 \\ \hline
P2C4S4A & P3C1S4 & 3.637364 \\ \hline
P1C1S1 & P4C5S2 & 3.6608 \\ \hline
\end{longtable}

Table 5.6 below shows the top results when longest common subsequence (LCS) is used and the symphony points were truncated to match each pair of symphonies. Table 5.7 shows the top results when LCS was used and no truncation was done. When truncation was done, a lot of top results became perfect matches as seen below and due to this, the results are deemed bad and inapproriate for comparison. The untruncated version shows better results with the top match being 61\%. Table 5.8, on the other hand, shows LCS results using the transitions only. This resulted in overall lower percentage matches but may still be a good metric and shall be evaluated further later on.

\begin{longtable}{|l|l|l|}
\caption{LCS Truncated Result}
\label{my-label}\\
\hline
Symphony A & Symphony B & Percent Match \\ \hline
\endfirsthead
%
\endhead
%
P1C1S3 & P1C4S4 & 100\% \\ \hline
P1C1S3 & P2C3S4A & 100\% \\ \hline
P1C1S3 & P2C4S1A & 100\% \\ \hline
P1C1S3 & P2C4S2A & 100\% \\ \hline
P1C1S3 & P2C4S3A & 100\% \\ \hline
P1C1S3 & P2C4S4A & 100\% \\ \hline
P1C1S3 & P2C5S1A & 100\% \\ \hline
P1C1S3 & P3C1S4 & 100\% \\ \hline
P1C1S3 & P3C1S5 & 100\% \\ \hline
P1C1S3 & P4C2S1A & 100\% \\ \hline
P1C1S3 & P4C4S1 & 100\% \\ \hline
P1C1S3 & P4C4S2 & 100\% \\ \hline
P1C1S3 & P5C2S3 & 100\% \\ \hline
P1C1S3 & P5C2S4A & 100\% \\ \hline
P1C1S3 & P5C3S4 & 100\% \\ \hline
P1C1S3 & P5C3S5A & 100\% \\ \hline
P1C1S3 & P5C5S1A & 100\% \\ \hline
P1C1S3 & P5C5S3 & 100\% \\ \hline
P1C1S3 & P5C5S4 & 100\% \\ \hline
P1C1S4 & P1C1S2 & 100\% \\ \hline
P1C1S4 & P1C2S5 & 100\% \\ \hline
P1C1S4 & P1C4S3 & 100\% \\ \hline
P1C1S4 & P1C4S5 & 100\% \\ \hline
P1C1S4 & P1C5S1 & 100\% \\ \hline
P1C1S4 & P2C1S1 & 100\% \\ \hline
P1C1S4 & P2C1S2 & 100\% \\ \hline
P1C1S4 & P2C1S3 & 100\% \\ \hline
P1C1S4 & P2C1S4 & 100\% \\ \hline
P1C1S4 & P2C2S2A & 100\% \\ \hline
P1C1S4 & P2C2S3A & 100\% \\ \hline
\end{longtable}

\begin{longtable}{|l|l|l|}
\caption{LCS Full Length Result}
\label{my-label}\\
\hline
Symphony A & Symphony B & Percent Match \\ \hline
\endfirsthead
%
\endhead
%
P2C4S4A & P2C4S4A & 61\% \\ \hline
P2C5S2A & P2C5S2A & 58\% \\ \hline
P2C5S4A & P5C2S2A & 56\% \\ \hline
P2C4S1A & P2C4S1A & 56\% \\ \hline
P2C4S4A & P5C3S5A & 55\% \\ \hline
P2C5S4A & P5C4S2A & 54\% \\ \hline
P2C4S4A & P5C5S3 & 53\% \\ \hline
P2C5S4A & P5C4S3A & 53\% \\ \hline
P2C5S4A & P5C5S5A & 51\% \\ \hline
P2C5S4A & P5C2S1A & 51\% \\ \hline
P2C5S4A & P5C4S4A & 49\% \\ \hline
P2C5S3A & P5C2S2A & 49\% \\ \hline
P2C4S1A & P5C5S3 & 49\% \\ \hline
P2C5S1A & P5C3S5A & 49\% \\ \hline
P2C5S4A & P5C3S1A & 49\% \\ \hline
P2C4S4A & P5C2S4A & 48\% \\ \hline
P2C5S4A & P5C3S2A & 48\% \\ \hline
P2C4S4A & P5C5S4 & 47\% \\ \hline
P2C1S4 & P5C5S3 & 47\% \\ \hline
P2C4S4A & P5C5S1A & 47\% \\ \hline
P2C5S4A & P5C3S3A & 47\% \\ \hline
P2C4S4A & P4C2S1A & 46\% \\ \hline
P2C1S4 & P5C3S3A & 46\% \\ \hline
P2C5S4A & P5C4S5A & 46\% \\ \hline
P2C1S4 & P5C2S5A & 46\% \\ \hline
P2C5S3A & P5C2S1A & 46\% \\ \hline
P2C1S4 & P5C3S2A & 45\% \\ \hline
P2C1S4 & P5C2S1A & 45\% \\ \hline
P2C5S4A & P5C4S1A & 45\% \\ \hline
P2C5S2A & P5C2S2A & 45\% \\ \hline
\end{longtable}

\begin{longtable}{|l|l|l|}
\caption{LCS  Transitions Result}
\label{my-label}\\
\hline
Symphony A & Symphony B & Percentage Match \\ \hline
\endfirsthead
%
\endhead
%
P2C5S4A & P5C2S2A & 42\% \\ \hline
P5C5S5A & P5C2S2A & 41\% \\ \hline
P5C4S3A & P5C5S5A & 41\% \\ \hline
P5C5S5A & P5C4S3A & 41\% \\ \hline
P5C2S1A & P5C2S2A & 41\% \\ \hline
P2C5S4A & P5C4S3A & 41\% \\ \hline
P5C4S2A & P5C5S5A & 40\% \\ \hline
P5C2S2A & P5C4S2A & 40\% \\ \hline
P5C2S2A & P5C4S3A & 40\% \\ \hline
P5C4S3A & P5C4S4A & 40\% \\ \hline
P2C5S4A & P5C4S2A & 40\% \\ \hline
P5C3S1A & P5C4S3A & 39\% \\ \hline
P5C2S2A & P5C3S2A & 39\% \\ \hline
P5C4S2A & P5C4S4A & 38\% \\ \hline
P5C4S3A & P5C5S1A & 38\% \\ \hline
P2C5S4A & P5C5S5A & 38\% \\ \hline
P5C3S2A & P5C4S3A & 37\% \\ \hline
P5C3S5A & P5C5S1A & 37\% \\ \hline
P2C5S4A & P5C2S1A & 37\% \\ \hline
P5C4S3A & P5C4S5A & 37\% \\ \hline
P5C1S4A & P5C1S5A & 37\% \\ \hline
P2C5S4A & P5C4S4A & 36\% \\ \hline
P5C3S5A & P5C5S3 & 36\% \\ \hline
P5C2S2A & P5C3S1A & 36\% \\ \hline
P5C4S4A & P5C5S5A & 36\% \\ \hline
P5C1S5A & P5C2S2A & 36\% \\ \hline
P5C2S2A & P5C4S4A & 36\% \\ \hline
P2C5S4A & P5C3S2A & 36\% \\ \hline
P2C5S4A & P5C3S1A & 35\% \\ \hline
P5C2S1A & P5C4S3A & 35\% \\ \hline
\end{longtable}

Table 5.9 below shows the top results for sequence matcher as was discussed in Chapter 4.

\begin{longtable}{|l|l|l|}
\caption{Sequence Match Result}
\label{my-label}\\
\hline
Symphony A & Symphony B & Percentage Match \\ \hline
\endfirsthead
%
\endhead
%
P1C3S1 & P1C5S2 & 83\% \\ \hline
P1C5S5 & P3C2S3 & 76\% \\ \hline
P1C5S2 & P1C5S4 & 74\% \\ \hline
P1C2S3 & P1C2S4 & 72\% \\ \hline
P4C3S2 & P3C2S1 & 69\% \\ \hline
P1C4S2 & P1C4S4 & 68\% \\ \hline
P1C2S4 & P3C2S2 & 68\% \\ \hline
P1C3S1 & P1C3S3 & 67\% \\ \hline
P3C2S2 & P1C2S4 & 67\% \\ \hline
P1C3S1 & P1C5S4 & 64\% \\ \hline
P3C3S2 & P3C2S1 & 63\% \\ \hline
P1C5S3 & P1C5S4 & 63\% \\ \hline
P1C5S5 & P3C2S2 & 62\% \\ \hline
P1C3S5 & P1C5S5 & 61\% \\ \hline
P1C2S1 & P1C5S5 & 61\% \\ \hline
P1C2S4 & P1C5S5 & 60\% \\ \hline
P1C2S4 & P1C2S1 & 60\% \\ \hline
P1C3S1 & P1C5S3 & 59\% \\ \hline
P1C2S1 & P1C2S4 & 59\% \\ \hline
P3C1S5 & P3C2S2 & 59\% \\ \hline
P3C4S5 & P1C4S2 & 59\% \\ \hline
P1C5S1 & P3C5S5 & 58\% \\ \hline
P3C2S3 & P1C4S2 & 58\% \\ \hline
P3C4S5 & P1C4S4 & 58\% \\ \hline
P3C2S3 & P3C5S5 & 57\% \\ \hline
P3C2S2 & P3C1S5 & 57\% \\ \hline
P3C3S5 & P3C2S1 & 57\% \\ \hline
P3C2S2 & P1C5S5 & 57\% \\ \hline
P1C5S2 & P1C5S3 & 57\% \\ \hline
P4C3S3 & P3C2S1 & 57\% \\ \hline
\end{longtable}

Levenshtein distance was applied to the cluster trajectories of the each symphony. Given the trajectory of the symphony, which is represented by letters from A to E, the researchers compared all of the symphonies and tabulated their respective Levenshtein distances. The same is done to the compressed version of these trajectories.  Table 5.10 shows the results of Levenshtein using the full length data while 5.11 shows the transitions results.

\begin{longtable}{|l|l|l|}
\caption{Levenshtein Result - Full Length}
\label{my-label}\\
\hline
Symphony A & Symphony B & Distance \\ \hline
\endfirsthead
%
\endhead
%
P1C1S4 & P1C5S2 & 95 \\ \hline
P1C3S3 & P1C5S4 & 102 \\ \hline
P1C1S3 & P3C5S5 & 107 \\ \hline
P1C1S5 & P1C5S2 & 114 \\ \hline
P1C1S4 & P1C1S5 & 115 \\ \hline
P1C1S5 & P1C5S3 & 125 \\ \hline
P1C2S1 & P1C3S5 & 147 \\ \hline
P1C3S5 & P1C5S5 & 155 \\ \hline
P1C2S2 & P1C5S1 & 167 \\ \hline
P1C1S4 & P1C4S1 & 180 \\ \hline
P1C2S1 & P1C5S5 & 184 \\ \hline
P1C3S3 & P1C5S2 & 188 \\ \hline
P1C1S3 & P1C4S2 & 194 \\ \hline
P1C5S2 & P1C5S4 & 202 \\ \hline
P1C1S3 & P1C3S1 & 205 \\ \hline
P1C5S2 & P1C5S3 & 205 \\ \hline
P1C3S1 & P3C5S5 & 207 \\ \hline
P1C2S3 & P1C3S1 & 217 \\ \hline
P1C1S4 & P1C5S3 & 219 \\ \hline
P1C3S5 & P3C1S2 & 223 \\ \hline
P1C4S1 & P1C5S2 & 226 \\ \hline
P1C1S4 & P1C3S3 & 228 \\ \hline
P1C2S1 & P1C5S4 & 234 \\ \hline
P1C3S3 & P2C2S1 & 234 \\ \hline
P1C3S5 & P1C4S1 & 238 \\ \hline
P1C2S4 & P1C3S2 & 240 \\ \hline
P1C3S3 & P1C4S1 & 242 \\ \hline
P1C4S2 & P3C5S5 & 244 \\ \hline
P1C5S4 & P2C2S1 & 249 \\ \hline
P1C2S1 & P1C3S3 & 252 \\ \hline
\end{longtable}

\begin{longtable}{|l|l|l|}
\caption{Levenshtein Transitions Result}
\label{my-label}\\
\hline
Symphony A & Symphony B & Distance \\ \hline
\endfirsthead
%
\endhead
%
P1C1S3 & P1C1S5 & 6 \\ \hline
P1C1S5 & P1C3S4 & 6 \\ \hline
P1C1S3 & P1C1S4 & 8 \\ \hline
P1C1S4 & P1C1S5 & 8 \\ \hline
P1C3S1 & P1C5S4 & 8 \\ \hline
P1C1S3 & P1C3S4 & 11 \\ \hline
P1C1S4 & P1C3S4 & 12 \\ \hline
P1C3S1 & P1C5S2 & 12 \\ \hline
P1C3S5 & P1C5S3 & 14 \\ \hline
P1C1S5 & P1C5S3 & 15 \\ \hline
P1C2S1 & P1C5S5 & 15 \\ \hline
P1C3S4 & P1C5S3 & 15 \\ \hline
P1C5S2 & P1C5S4 & 15 \\ \hline
P1C1S3 & P1C5S3 & 16 \\ \hline
P1C1S3 & P1C3S5 & 17 \\ \hline
P1C1S4 & P1C5S3 & 20 \\ \hline
P1C3S1 & P1C5S3 & 21 \\ \hline
P1C1S5 & P1C3S5 & 22 \\ \hline
P1C2S1 & P1C2S3 & 22 \\ \hline
P1C5S3 & P1C5S4 & 22 \\ \hline
P1C1S4 & P1C3S5 & 23 \\ \hline
P1C3S4 & P1C3S5 & 24 \\ \hline
P1C2S1 & P3C2S3 & 25 \\ \hline
P1C2S3 & P3C2S3 & 25 \\ \hline
P1C3S5 & P1C5S4 & 25 \\ \hline
P1C2S2 & P1C5S2 & 27 \\ \hline
P1C2S3 & P1C2S4 & 27 \\ \hline
P1C5S5 & P3C2S3 & 27 \\ \hline
P1C3S1 & P1C3S5 & 28 \\ \hline
P1C5S2 & P1C5S5 & 28 \\ \hline
\end{longtable}

In order to verify which metric was the best, we compare the top 30 most similar pairs of symphonies generated by each metric and find out which two metrics had the most matches. Note also that we chose to include different implementations for each algorithm like the mean procedure of Manhattan distance over the summation procedure. Based on Table 5.12 below, mean and summation procedure of Manhattan distance would produce non-similar results since only 10 out of 30 matched. 

Among the different metrics, Levenshtein and sequence match matched best with 8 out of 30 matches, although the score is still low. Manhattan distance and Edit Distance had the most number of entries in the top 10, with Manhattan distance sum having 5 entries while the mean version having 3. Edit Distance has a total of 3 for the full length and 3 for the transitions. LCS had the worst performance with all its entries below the top 10 and sequence match performed decent with half of its entries above the top 10 and the rest below.

There resulted a total of 12 entries for the non-transitions and 8 for transitions which means that testing only the transitions for comparison is not enough to find similarities as its results are more inconsistent compared to that of the original length.

\begin{longtable}[c]{|l|l|l|}
\caption{Metric Evaluation from Pairwise Symphonies}
\label{my-label}\\
\hline
Metric 1 & Metric 2 & Matches \\ \hline
\endfirsthead
%
\endhead
%
Manhattan Distance Mean & Manhattan Distance Sum & 10 \\ \hline
Manhattan Distance Mean & LCS Truncated & 1 \\ \hline
Manhattan Distance Mean & Sequence Match & 1 \\ \hline
Manhattan Distance Sum & LCS Truncated & 0 \\ \hline
Manhattan Distance Sum & Sequence Match & 0 \\ \hline
LCS Truncated & Sequence Match & 0 \\ \hline
Manhattan Distance Transition Sum & Manhattan Distance Mean & 2 \\ \hline
Manhattan Distance Transition Sum & Manhattan Distance Sum & 3 \\ \hline
Manhattan Distance Transition Sum & LCS Truncated & 0 \\ \hline
Manhattan Distance Transition Sum & Sequence Match & 1 \\ \hline
Edit Distance Full & Manhattan Distance Mean & 5 \\ \hline
Edit Distance Full & Manhattan Distance Sum & 6 \\ \hline
Edit Distance Full & LCS Truncated & 0 \\ \hline
Edit Distance Full & Sequence Match & 4 \\ \hline
Edit Distance Full & Manhattan Distance Transition Sum & 2 \\ \hline
Edit Distance Transitions & Manhattan Distance Mean & 3 \\ \hline
Edit Distance Transitions & Manhattan Distance Sum & 3 \\ \hline
Edit Distance Transitions & LCS Truncated & 0 \\ \hline
Edit Distance Transitions & Sequence Match & 8 \\ \hline
Edit Distance Transitions & Manhattan Distance Transition Sum & 2 \\ \hline
Edit Distance Transitions & Edit Distance Full & 5 \\ \hline
\end{longtable}

In order to evaluate which composers had a wider impact to other symphonies, we tabulate the average distances or percentage per composer per era to see which composer had the least distance or the highest percentage. If different metrics produce similar results, then we can say that those metrics produced good results. Table 5.13 shows the average number of edits per composer, under each era. The table shows that Gabrieli (P1C3) has the lowest average Levenshtein distance from all of the composers across all eras while Rachmaninoff (P5C2) has the largest. It can be seen that composers from the 20th Century era (P5) returned the largest results, this could be due to the fact that the symphonies are longer than the other symphonies, which results to them having a longer trajectory string. In order to address the issue, the researchers have normalized the Levenshtein distances by dividing the results by the length of the longer trajectory string.

\begin{longtable}{|l|l|l|l|l|l|}
\caption{Average Levenshtein Distance per Composer}
\label{my-label}\\
\hline
Row & Baroque & Classical & 19th Century & Romantic & 20th Century \\ \hline
\endfirsthead
%
\endhead
%
C1 & 1191.4 & 1120.7 & 1420.6 & 3097.2 & 2447.4 \\ \hline
C2 & 724.5 & 1204 & 1309.1 & 2232.6 & 4005.6 \\ \hline
C3 & 468.3 & 815.3 & 1424.7 & 3308.7 & 3380.3 \\ \hline
C4 & 1018 & 1755.8 & 1639.6 & 2451 & 3133.3 \\ \hline
C5 & 503 & 2378.3 & 1852.2 & 1559.4 & 3863 \\ \hline
\end{longtable}

\begin{longtable}{|l|l|l|l|l|l|}
\caption{Average Normalized Levenshtein Distance per Composer}
\label{my-label}\\
\hline
Row & Baroque & Classical & 19th Century & Romantic & 20th Century \\ \hline
\endfirsthead
%
\endhead
%
C1 & 0.854 & 0.439 & 0.693 & 0.705 & 0.436 \\ \hline
C2 & 0.743 & 0.681 & 0.721 & 0.701 & 0.66-1 \\ \hline
C3 & 0.857 & 0.538 & 0.442 & 0.7101 & 0.611 \\ \hline
C4 & 0.781 & 0.569 & 0.635 & 0.755 & 0.569 \\ \hline
C5 & 0.736 & 0.63 & 0.717 & 0.7 & 0.649 \\ \hline
\end{longtable}

The researchers experimented on compressing the trajectory string, only taking into consideration the different transitions from letter to letter, ignoring the number of times the letter was repeated. This compressed trajectory string will now be compared using Levenshtein distance. Table 5.15 shows the average Levenshtein distances per composer using the compressed trajectory string. Viadana (P1C5) has the lowest average Levenshtein distance.

\begin{longtable}{|l|l|l|l|l|l|}
\caption{Average Levenshtein Distance (Compressed) per Composer}
\label{my-label}\\
\hline
Row & Baroque & Classical & 19th Century & Romantic & 20th Century \\ \hline
\endfirsthead
%
\endhead
%
C1 & 160.7 & 422.6 & 229.5 & 852.3 & 702.3 \\ \hline
C2 & 272.1 & 623 & 199.1 & 674.4 & 1736.7 \\ \hline
C3 & 111.2 & 191.7 & 509.7 & 260.5 & 1547.7 \\ \hline
C4 & 228.3 & 795.6 & 290.1 & 428.1 & 1583.2 \\ \hline
C5 & 40.1 & 1448.3 & 403.2 & 302.1 & 1581.6 \\ \hline
\end{longtable}

The compressed trajectory strings also underwent normalized Levenshtein distance and results are shown in Table 5.16. Antheil (P5C1) is shown to have the smallest average Levenshtein distance. It also can be noted that having a low Levenshtein distance does not not guarantee having a low normalized Levenshtein distance. In this case, 

\begin{longtable}{|l|l|l|l|l|l|}
\caption{Average Normalized Levenshtein Distance (Compressed) per Composer}
\label{my-label}\\
\hline
Row & Baroque & Classical & 19th Century & Romantic & 20th Century \\ \hline
\endfirsthead
%
\endhead
%
C1 & 0.873 & 0.58 & 0.624 & 0.593 & 0.319 \\ \hline
C2 & 0.697 & 0.641 & 0.693 & 0.54 & 0.621 \\ \hline
C3 & 0.807 & 0.382 & 0.579 & 0.626 & 0.532 \\ \hline
C4 & 0.64 & 0.485 & 0.54 & 0.722 & 0.492 \\ \hline
C5 & 0.559 & 0.565 & 0.676 & 0.584 & 0.544 \\ \hline
\end{longtable}

Similarly, the result from Manhattan distance was tabulated to show the average distances of each composer per era as seen in Table 5.17. The result shows that Gabrieli of Baroque era (P1C3) had the smallest distance value using the Manhattan while Schubert of Romantic era (P4C3) had the largest. This means that Gabrieli had symphonies that are closer to more symphonies than by other composers while Schubert had symphonies that are more dissimilar to other symphonies. Baroque era, in general, also had overall smaller distances compared to other eras while Romantic era had the largest.

\begin{longtable}{|l|l|l|l|l|l|}
\caption{Average Manhattan Distance per Composer}
\label{my-label}\\
\hline
Row & Baroque & Classical & 19th Century & Romantic & 20th Century \\ \hline
\endfirsthead
%
\endhead
%
C1 & 8166.221 & 15215.32 & 10140.51 & 15769.45 & 23068.61 \\ \hline
C2 & 6089.764 & 8331.856 & 13083.8 & 14746.23 & 19661.5 \\ \hline
C3 & 4140.392 & 9954.428 & 17667.8 & 27076.32 & 18392.58 \\ \hline
C4 & 7199.323 & 16218.67 & 11979.14 & 19391.64 & 20215.49 \\ \hline
C5 & 4918.109 & 17404.75 & 12183.51 & 13561.83 & 22579.82 \\ \hline
\end{longtable}

Using the normalized Manhattan distance results with the same procedure as above, it can be seen in Table 5.18 that this time, Viadana of Baroque era (P1C5) had the smallest distance, although Gabrieli, the smallest distance from the non-normalized manhattan distance result, still had a very small distance compared to most composers. This time, however, Antheil of 20th Century had the largest distance instead of Schubert

\begin{longtable}{|l|l|l|l|l|l|}
\caption{Average Normalized Manhattan Distance per Composer}
\label{my-label}\\
\hline
Row & Baroque & Classical & 19th Century & Romantic & 20th Century \\ \hline
\endfirsthead
%
\endhead
%
C1 & 491.3077674 & 2117.378009 & 1291.064646 & 3230.844887 & 6138.347886 \\ \hline
C2 & 905.0273959 & 2499.334816 & 1299.300606 & 3090.077958 & 4774.202967 \\ \hline
C3 & 461.8045741 & 2436.608763 & 2771.299898 & 2163.809725 & 4931.113653 \\ \hline
C4 & 1279.07484 & 3778.909272 & 1838.724787 & 2024.689936 & 5613.785691 \\ \hline
C5 & 410.6783212 & 4618.480295 & 1784.439225 & 1848.743024 & 5857.821717 \\ \hline
\end{longtable}