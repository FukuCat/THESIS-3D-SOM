\chapter{Conclusion}
The results of t-SNE produced good visualizations that represented each symphony, composer, and era well. The result of using t-SNE provided a good substitute for SOM as its results had the same characteristics while also being fast and almost the same overall visual maps. The colored maps, however, did not improve upon the previous visualization that SOMphony has provided, since the points still overlap each other as time progresses. Aside from this, using visual results is not enough to compare two different symphonies, there is still the need to support the visual observations by quantitative metrics. 

Among the metrics tested for comparison of symphonies, both Manhattan distance and Levenshtein distance provided results that also had a high similarity percentage when plotted in the 3D visualization. Both these metrics\' 2D symphony graphs did not look anything alike however. LCS, on the other hand, had symphonies that look very much alike in the 2D graph but have a lower similarity percentage in the 3D graph. We can conclude here that even though t-SNE produced a 2D graph that looked similar, incorporating time using the 3D graph will not necessarily give the same similarity as with the 2D. Time sequence plays a huge role in determining whether a symphony is similar with another. For instance, if a symphony starts from the right side and moves to the left part of the 2D graph as it progresses through time, it would look just the same as a symphony that started from the left and ended at the right.

The top matches from LCS and from Levenshtein distance had symphonies that were composed far apart from each other while the result of Manhattan distance for both the original and the compressed data had symphonies that were composed near each other. For comparing symphonies, date of composition therefore will not dictate whether a symphony is similar or not depending on the proximity of the dates. Two symphonies that are similar have a hgih chance of being composed near each other but two symphonies that were composed far apart are not necessarily non-similar. Factors such as a composer's influence through the years plays a huge role in the similarity of symphonies because people learn music through former musicians\' compositions as well.

When the composers were all quantized as to how similar their symphonies are with each other, both Antheil, which resulted as the least distance from Levenshtein, and Gossec, which resulted as the least distance from Manhattan, had symphonies that looked very similar with each other. Additionally, both composers\' symphonies were composed very near each other.

From this, we can conclude that since both Levenshtein distance and Manhattan distance produced good results in general, both these metrics are good quantifiers for comparing the similarity of two symphonies or maybe not just music, but any data.

Prior to the research, it was hypothesized that using simply the transitions instead of all the points might also produce good comparisons and indeed as shown in Chapter 5's results, both the original and the compressed versions had generally the same results.

Even though this research’s results point to these conclusions, it does not mean that the metrics that were deemed bad are always bad when comparing music in general quantitatively. Some  factors pour in such as the research methodology followed in this research. It may have affected the result either positively or poorly. For future works, it is highly encouraged to either alter the research methodology or test plenty more metrics to further test music comparisons and see if other metrics can perform better than the metrics tested here. Testing different datasets and using the same methodology might also be a possible research work in the future as this methodology should also work for other datasets and not just music.

\nocite{*}