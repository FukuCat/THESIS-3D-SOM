\chapter{Conclusion}
The results of t-SNE produced good visualizations that represented each symphony, composer, era well. The result of using t-SNE provided a good substitute for SOM as its results had the same characteristics while also being fast and almost the same overall visual maps. The colored maps, however, did not improve upon the previous visualization that SOMphony has provided, since the points still overlap each other as time progresses. Aside from this, using visual results is not enough to compare two different symphonies, there is still the need to support the visual observations by quantitative metrics. 

Among the metrics tested for comparison of symphonies in Table 5.12 , two metrics had the most overall matches with other metrics as shown in Chapter 5. Both the Manhattan distance and Levenshtein (Edit-Distance) had the most top matches; therefore, we can conclude that among the metrics tested, both these metrics produced the best results. Longest Common Subsequence performed the worst no matter what methodology was used, truncated, full-length, or transitions. Sequence match performed average as its entries are split above and below the top ten.

Prior to the research, it was hypothesized that using simply the transitions instead of all the points might also produce good comparisons; however, as shown in Chapter 5’s Table 5.12, majority of the entries above the top ten are the non-transition tests. We cannot say however that comparing with transitions is generally bad, some transition tests in the metrics also produced good results even if they are in the minority.

Even though this research’s results point to these conclusions, it does not mean that the metrics that were deemed bad are always bad when comparing music in general quantitatively. Some  factors pour in such as the research methodology followed in this research may have affected the result either positively or poorly. For future works, it is highly encouraged to either alter the research methodology or test plenty more metrics to further test music comparisons.

\nocite{*}