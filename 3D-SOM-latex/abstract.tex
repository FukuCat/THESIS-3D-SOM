%%%%%%%%%%%%%%%%%%%%%%%%%%%%%%%%%%%%%%%%%%%%%%%%%%%%%%%%%%%%%%%%%%%%%%%%%%%%%%%%%%%%%%%%%%%%%%%%%%%%%%
%
%   Filename    : abstract.tex 
%
%   Description : This file will contain your abstract.
%                 
%%%%%%%%%%%%%%%%%%%%%%%%%%%%%%%%%%%%%%%%%%%%%%%%%%%%%%%%%%%%%%%%%%%%%%%%%%%%%%%%%%%%%%%%%%%%%%%%%%%%%%

\begin{abstract}
Symphonies are musical compositions for orchestras that consist of several large sections called movements. There are five major musical periods namely the Baroque Period, Classical Period, 19th Century, Romantic Period, and the 20th Century that played a big role during the height of symphonies. This research will compare pairs of symphonies to determine their similarity, and create a visualization method to represent the comparison. From a previous research work by Azcarraga \& Flores (2016) that compared symphonies though self-organizing maps (SOM),  this research work will compare symphonies through visualization in a 3D SOM. By having a visual representation, the research provides an interactive and straightforward way to identify which parts of the symphonies are most similar and by adding the concept of time series on the clustering of the 3D SOM, more accurate results can be made. Quantitative data will be gathered through cluster analysis and using Euclidean distance to measure the musical trajectories of each musical segment of the symphony to produce the overall 3D SOM. T-SNE will also be experimented upon to see if it also produces optimal visualization for symphonies like with SOM.


\begin{flushleft}
\begin{tabular}{lp{4.25in}}
\hspace{-0.5em}\textbf{Keywords:}\hspace{0.25em} & Machine Learning, Music, Time Dimension, Self-Organizing Map, K-Means Clustering.\\
\end{tabular}
\end{flushleft}
\end{abstract}
