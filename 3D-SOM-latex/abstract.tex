%%%%%%%%%%%%%%%%%%%%%%%%%%%%%%%%%%%%%%%%%%%%%%%%%%%%%%%%%%%%%%%%%%%%%%%%%%%%%%%%%%%%%%%%%%%%%%%%%%%%%%
%
%   Filename    : abstract.tex 
%
%   Description : This file will contain your abstract.
%                 
%%%%%%%%%%%%%%%%%%%%%%%%%%%%%%%%%%%%%%%%%%%%%%%%%%%%%%%%%%%%%%%%%%%%%%%%%%%%%%%%%%%%%%%%%%%%%%%%%%%%%%

\begin{abstract}
Symphonies are musical compositions for orchestras usually consisting of several large sections called movements. They are usually composed of three to five movements, depending on the time period and are constructed by many different composers (Libin, Laurence, 2014). There are five major musical periods namely the Baroque period, Classical Period, 19th Century, Romantic Period, and the 20th Century. By visualizing and determining the relationship of one composition to another, this research will be able to show the visual representation of the styles of the different composers. Similarly, the research will also help identify the influence of composers on one another from one musical period to the next. Azcarraga \& Flores (2016)’s study tried to understand their relationship using machine learning, which uses self-organizing maps (SOM) and K-Means to determine clusters, and used the frequency counts in order to determine the comparison, which resulted into a visually comparable image of trajectories. Using frequency count however does not take into account the sequence as well as transitions of music from one after the other. This research aims to use the concept of time series on the clustering of the self-organizing maps. By applying time series on the maps, more accurate results can be made.

\begin{flushleft}
\begin{tabular}{lp{4.25in}}
\hspace{-0.5em}\textbf{Keywords:}\hspace{0.25em} & Machine Learning, Music, Time Seires, Self-Organizing Map, K-Maps.\\
\end{tabular}
\end{flushleft}
\end{abstract}
