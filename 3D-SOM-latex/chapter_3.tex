%%%%%%%%%%%%%%%%%%%%%%%%%%%%%%%%%%%%%%%%%%%%%%%%%%%%%%%%%%%%%%%%%%%%%%%%%%%%%%%%%%%%%%%%%%%%%%%%%%%%%%
%
%   Filename    : chapter_3.tex 
%
%   Description : This file will contain your Research Methodology.
%                 
%%%%%%%%%%%%%%%%%%%%%%%%%%%%%%%%%%%%%%%%%%%%%%%%%%%%%%%%%%%%%%%%%%%%%%%%%%%%%%%%%%%%%%%%%%%%%%%%%%%%%%

\chapter{Research Methodology}
This chapter contains phases and activities that will be performed to accomplish the research. The phases listed here will be arranged sequentially unless otherwise stated.

\section{Research Activities}
\subsection{Concept Formulation and Review of Related Literature}
This phase will concern the consolidation of the thesis requirements such as the objective of the research, the research problem to be tackled, and the scopes and limitations of such research. Literatures related to 2D and 3D self-organizing maps, music feature classifications, k-means clustering, and  machine learning will be reviewed as part of the Review of Related Literature.

\subsection{Data Gathering}

This phase will concern the  gathering of the additional symphonies to be used for the research. Aside from the music dataset used in the research of Azcarraga \& Flores (2016), 2 symphonies will be added to each composer, summing up to a total of 75 symphonies. This phase will also include the division of the additional symphonies into 1 sec music segments in preparation for music feature extraction. 


\subsection{Development}
In this phase, the concept of time series will be incorporated to the algorithm for training the SOM. The system for the visual representation of the trained 3D SOM’s will also be developed. This 3D SOM visualization will look similar to Azcarraga, A., Caronongan, A., Setiono, R., \& Manalili, S. (2016)’s research work but with the altered algorithm for time series.
\subsection{Training}
In this phase, training of the symphonies into 3D SOM’s will be done. JAudio will still be used for the music feature extraction and algorithm will for clustering will be modified accordingly based on the results of development. 600 optimal features among all the features will be selected as a result. The newly added, together with the old symphonies should be trained all over again using the new 3D SOM.

\subsection{Performance Evaluation and Human Evaluation}
In this phase, comparison between the results of the previous 2D SOM and the results of the 3D SOM for the symphonies from the old dataset will be made. This, together with the results of the symphonies from the new dataset will make up the quantitative comparison of symphonies. This phase will also include the evaluation of results from at least 50 human participants with knowledge about music who will listen and give comparisons on specified symphonies. This will make up the qualitative comparison of symphony. Both the quantitative and qualitative comparisons will be used in analyzing and concluding the results of the experiment.
\subsection{Documentation}
This phase will be done all throughout the whole research timeframe. This will include taking down notes on observations and findings during experiments and during the review of related literature, writing related technical documents, and the research paper itself.


\subsection{Calendar of Activities}
Table 3.1 shows the time table for the activities involved with the research. The numbers represent the number of weeks worth of activity. The ♫ symbol represents the number of weeks allotted for the month.