%%%%%%%%%%%%%%%%%%%%%%%%%%%%%%%%%%%%%%%%%%%%%%%%%%%%%%%%%%%%%%%%%%%%%%%%%%%%%%%%%%%%%%%%%%%%%%%%%%%%%%
%
%   Filename    : chapter_3.tex 
%
%   Description : This file will contain your Research Methodology.
%                 
%%%%%%%%%%%%%%%%%%%%%%%%%%%%%%%%%%%%%%%%%%%%%%%%%%%%%%%%%%%%%%%%%%%%%%%%%%%%%%%%%%%%%%%%%%%%%%%%%%%%%%

\chapter{Research Methodology}
This chapter contains phases and activities that will be performed to accomplish the research. The phases listed here will be arranged sequentially unless otherwise stated.

\section{Research Activities}
\subsection{Concept Formulation and Review of Related Literature}
This phase will concern the consolidation of the thesis requirements such as the objective of the research, the research problem to be tackled, and the scopes and limitations of such research. Literatures related to 2D and 3D self-organizing maps, music feature classifications, k-means clustering, and  machine learning will be part of the Review of Related Literature.

\subsection{Data Gathering}

This phase will concern the  gathering of the additional symphonies to be used for the research to provide more reliable results. Aside from the music dataset used in SOMphony, 2 symphonies will be added to each composer, summing up to a total of 5 symphonies per composer and 125 in all. The proponents have decided to maintain 5 symphonies per composer so that the data set will be balanced. The process of selecting which symphonies to be added would be by random to have a better grasp of the general style of the composer. The audio files would be retrieved from online sources. The researchers  would not take into consideration the file type and bitrate of the audio files since music data that is free for use is limited. 


\subsection{Pre-processing}
To start pre-processing,  the audio files would be converted into wav files in preparation for splitting. WaveSplitter will be used in splitting the audio file into 1 second segments at intervals of 0.5 second. These segments would undergo feature extraction using jAudio. The result would be an xml file containing all the features determined for each segment. The researchers would then run RegEx to extract the unnecessary text in preparation for labeling. Since the proponents would have supervised learning, the data needs to be labelled according to their composer, composition and file name. 

\subsection{Feature Selection}
In this phase, the proponents will trim down the 600 features that jAudio has extracted. Using decision trees, the top 20 nodes will be selected as the top 20 features. The proponents have decided to have 20 as an arbitrary value for the features to be used. By doing feature selection, the data set would have a uniform number of features for all symphonies and it would also enhance the efficiency of training the SOM. The tree model produced after feature selection is what would be used in training the SOM.

\subsection{Visualization}
In this phase, the proponents will trim down the 600 features that jAudio has extracted. Using decision trees, the top 20 nodes will be selected as the top 20 features. The proponents have decided to have 20 as an arbitrary value for the features to be used. By doing feature selection, the data set would have a uniform number of features for all symphonies and it would also enhance the efficiency of training the SOM. The tree model produced after feature selection is what would be used in training the SOM.

\subsection{Performance Evaluation and Human Evaluation}
In this phase, the proponents limit themselves to 50 participants. The participant profile would be people that have experience with classical music. In the case that the target amount is not reached within two months, the researchers will proceed to analyze the results they have. The participants would be presented with a 3D graph and two music players. They are tasked to annotate specific regions of the symphony and verify if they are indeed similar. However we do not limit the participants to the specified regions, the participants are free to annotate parts that they believe sound similar. 

\subsection{Documentation}
This phase will be done all throughout the whole research timeframe. This will include taking down notes on observations and findings during experiments and during the review of related literature, writing related technical documents, and the research paper itself.


\subsection{Calendar of Activities}
Table 3.1 shows the time table for the activities involved with the research. The numbers represent the number of weeks worth of activity. The ♫ symbol represents the number of weeks allotted for the month.

\nocite{Dubnov}
\nocite{Azcarraga2016}
\nocite{cambouropoulosEmilios}
\nocite{3dsom}
\nocite{correa}
\nocite{imogen}
\nocite{libin}
\nocite{foote}
\nocite{silla}
\nocite{mcfee}
\nocite{hepokoski}