%%%%%%%%%%%%%%%%%%%%%%%%%%%%%%%%%%%%%%%%%%%%%%%%%%%%%%%%%%%%%%%%%%%%%%%%%%%%%%%%%%%%%%%%%%%%%%%%%%%%%%
%
%   Filename    : chapter_1.tex 
%
%   Description : This file will contain your Research Description.
%                 
%%%%%%%%%%%%%%%%%%%%%%%%%%%%%%%%%%%%%%%%%%%%%%%%%%%%%%%%%%%%%%%%%%%%%%%%%%%%%%%%%%%%%%%%%%%%%%%%%%%%%%

\chapter{Research Description}
\label{sec:researchdesc}    %--note: labels help you with hyperlink editing (using your IDE)

\section{Overview of the Current State of Technology}
\label{sec:overview}
Music has been a part of our culture for hundreds of years, classical music being one of the oldest genre of music. Classical music is rooted in the traditions of early western music and to this day, many people refer to classical music as serious music. Musicians, however, use classical music to refer to music composed during 1750 to 1825, otherwise known as the Classical Era \cite{bernstein}. The central norms of classical music became established between 1550 and 1900, which is known as the common-practice period. The common-practice period contains a bulk of what we now know as classical music. Under this period there are 3 musical eras: Baroque, Classical and Romantic. Music from the Baroque period are decorated and elaborate, with little to no expression. Works from the Classical era contain repetitive dynamics and clean transitions. In contrast to music from the Baroque period, music from the Romantic period are expressive and emotive, having the ability to paint a vivid picture in the minds of the listeners (Grout, Palisca, 1996); however, Dahlhaus (1981) points out that another musical era existed between the Classical and the Romantic period and he refers to this as the 19th century era. This era serves as the transition period for classical and the romantic period, thus having similarities in style with both eras. After the common-practice period comes the 20th century era, which explores modernism, impressionism, neoclassicism and experimental music. 

It was in the common-practice era when symphonies began to be composed. Libin (2014) describes symphonies as lengthy forms of musical compositions which are almost always written for orchestras and are consisting of several large movements. With a history of almost 300 years, symphonies today are viewed as the very pinnacle of classical music where Beethoven, Brahms, Mozart and other renowned composers were able to find a venue for transcending their creativities and overall influencing them heavily on their music. During the course of the 18th century, the tradition was  to write four-movement symphonies (Hepokoski \& Darcy, 2006).

Throughout time, different styles have developed, each having features unique to themselves. Tilden (2013) notes the historical influence of composers with each other and how similar the methods of composing classical music are with pop music.  Due to these facts presented, symphonies written in the early 20th century may be influenced by the great composers and compositions of the previous eras. Analyzing these musical relationships and comparing one to another is a research area that could be done through both manual and machine learning methods.

McFee, Barringtong \& Lanckriet (2012) compare the usage of context-based manual semantic annotation versus their proposed optimized content-based similarity learning framework. With machine learning, the usage of high-quality training data without active user participation and the analysis of more data is possible than with feedback or survey data from active user participation. Human error in the analysis process can also be minimized with machine learning since human-supervised training is minimal. Corrêa \& Rodrigues (2016) shows the analysis of music features using machine learning techniques. According to the MIR community (Silla \& Freitas, 2009), the two main representation of music feature content are either audio-recorded or symbolic-based. The former employs the explicit recording of audio files while the latter uses symbolic data files such as MIDI or KERN.

SOMphony, a research paper by Azcarraga \& Flores (2016) aims to understand the relationship of compositions between the same composer to denote style as well as to determine if there are similarities between compositions of different periods of music to denote influence between time periods. The research showed the relationships and influences between composers from 5 major musical periods, namely the Baroque Period, Classical Period, 19th Century music, Romantic Period and the 20th century. The research focuses on self-organizing maps (SOM) that are trained using 1-second music segments extracted from the 45 different symphonies. The trained SOM is then further processed by doing a k-means clustering of the node vectors, allowing quantitative comparison music trajectories between symphonies. Their research showed that using self-organizing maps are indeed helpful in visualizing the musical features of a symphony, making it easier to create insights about the relationships within the different pieces and composers. The research concludes that a larger dataset would be needed to confirm whether the approach is indeed valid. 

However, SOMphony does not take into consideration the notion of time. In time series data, each instance represents a different time step and the attributes give values associated with that time (Witten  \& Frank, 2005).  To be able to generate time sensitive musical analysis, time series is to be added to the SOM and a new visualization in 3D space would need to be created.

\section{Research Objectives}
\label{sec:researchobjectives}
\subsection{General Objective}
\label{sec:generalobjective}
To incorporate the use of time series in the visualization of symphonies for comparison of similarities
\subsection{Specific Objectives}
\label{sec:specificobjectives}
\begin{enumerate}
\item To include more musical pieces to the data set;
\item To perform feature selection to determine optimal features to be used;
\item To add in the time series variable;
\item To create a 3D visualization model for the data;
\item To have participants listen and annotate the musical pieces for qualitative data;
\item To verify the results of the 3D SOMphony through the results obtained from the human participants;
\end{enumerate}
\section{Scope and Limitations of the Research}
\label{sec:scopelimitations}
To expand the data set of SOMphony, the proponents will add an additional 2 symphonies to the existing 3 symphonies for each composer. This will result to a total of 125 symphonies all in all. By having an equal number of symphonies per composer, this maintains a balanced data set for all composers. The criteria for choosing the symphonies to be added would be random to have a better grasp on the general style of the composer.

To be able to generate a self-organizing map, the proponents will use jAudio to extract 600 audio features from musical segments generated from the symphony. For the labelling phase, the proponents will classify the selected features by composition because the research focuses on comparing different compositions and comparing eras or composers. The 600 audio features would be trimmed down through feature selection. Decision trees would be used to determine the features to be kept since those nodes that are nearest to the root node of the tree would be dubbed as the more important features compared to the others. The proponents have decided to have 20 as an arbitrary value for the features to be used. By selecting only 20 features, the data set would have a uniform number of features for all symphonies and it would also enhance the efficiency of training the SOMs as it would not take a long time to extract 20 features compared to extracting 600. 

In incorporating the time series, the musical piece is divided into 1 second segments in order to be uniform all throughout the piece and to avoid incomplete notes. A 0.5 second overlap is used to be able to consider transitions between each second. 

To create a 3D model to represent the symphony, the proponents will assign each generated SOM to a point in time and will be used to create a graph representing each map in a time series. As a result of using time series, our research will be able to better differentiate symphonies that use similar themes but at different periods of time in the composition.

In gathering qualitative data from human participants, the proponents limit themselves to 50 participants. In the case that the target amount is not reached within two months, the researchers will proceed to analyze the results they have obtained. The participant profile would be people that have experience or familiarity with classical music. The participants would be presented with a 3D graph and two music players. They are tasked to annotate specific regions of the symphony if they are indeed similar. However we do not limit the participants to the specified regions, the participants are free to annotate parts that they believe sound similar. 

Similar to SOMphony, the proponents will  focus on representation of symphonies using SOMs for the purpose of comparison to other symphonies. Through the data obtained from the human participants, the proponents will be able to validate if the 3D visualization method is enough to represent the entirety of the symphony for comparison.


\section{Significance of the Research}
\label{sec:significance}
As Tilden (2013) states, the structure of classical and modern music are very similar, having the verse-chorus structure and modern pop songs are first composed instrumentally as similar with how classical music is composed. Modern music just takes classical music further by adding in voice and combining the different techniques employed by classical music. As this study focuses on comparing different symphonies and analyzing to see how similar they are, the results of this study will show us trends among composers in terms of their influence on one another in a musical era, the influence one composer had over other composers from a later era, and what the particular style of a composer would look like in the SOM. This research will show whether composers from back then had a lasting influence on music 100 or so years from a particular composer’s time period. This research can also show if a particular composer has a definite coherent style that is present in his musical pieces by comparing his works.

Some possible future application of the results of this study would include the improvement of existing music information retrieval (MIR) techniques used by music databases. Corrêa, D. C., \& Rodrigues, F. A. (2016)’s research shows a possible improvement on automatic music genre classification using symbolic-based music features. Similarly, this research can also be used to further improve the algorithms used by playlist managers for the retrieval of similar songs from music databases using the comparison of the trained SOMs.

The application of time series in machine learning would benefit studies outside of music that incorporates the use of time sensitive data. It can be used in future works regarding traffic modelling, weather monitoring, prediction, and other time sensitive fields.

