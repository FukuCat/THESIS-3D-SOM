%%%%%%%%%%%%%%%%%%%%%%%%%%%%%%%%%%%%%%%%%%%%%%%%%%%%%%%%%%%%%%%%%%%%%%%%%%%%%%%%%%%%%%%%%%%%%%%%%%%%%%
%
%   Filename    : chapter_1.tex 
%
%   Description : This file will contain your Research Description.
%                 
%%%%%%%%%%%%%%%%%%%%%%%%%%%%%%%%%%%%%%%%%%%%%%%%%%%%%%%%%%%%%%%%%%%%%%%%%%%%%%%%%%%%%%%%%%%%%%%%%%%%%%

\chapter{Research Description}
\label{sec:researchdesc}    %--note: labels help you with hyperlink editing (using your IDE)

\section{Overview of the Current State of Technology}
\label{sec:overview}
	Music has been a part of people’s culture for hundreds of years, classical music being one of the oldest genres of music. Classical music is rooted in the traditions of early western music and to this day, many people refer to classical music as serious music. Musicians, however, use classical music to refer to music composed during 1750 to 1825, otherwise known as the Classical Era (Bernstein, 1959). The central norms of classical music became established between 1550 and 1900, which is known as the common-practice period. The common-practice period contains the majority of what we now know as classical music. Under this period there are 3 musical eras: Baroque, Classical and Romantic. Music from the Baroque period are decorated and elaborate, with little to no expression. Works from the Classical era contain repetitive dynamics and clean transitions. In contrast to music from the Baroque period, music from the Romantic period are expressive and emotive, having the ability to paint a vivid picture in the minds of the listeners (Grout \& Palisca, 1996); however, Dahlhaus (1981) points out that another musical era existed between the Classical and the Romantic period and he refers to this as the 19th century era. This era serves as the transition period for the classical and romantic period, thus having similarities in style with both eras. After the common-practice period comes the 20th century era, which explores modernism, impressionism, neoclassicism and experimental music. 

It was in the common-practice era when symphonies began to be composed. Libin (2014) describes symphonies as lengthy forms of musical compositions which are almost always written for orchestras and are consisting of several large movements. They are composed of three to five movements, depending on the time period and are constructed by many different composers (Libin, 2014). There are five major musical periods namely the Baroque Period, Classical Period, 19th Century, Romantic Period, and the 20th Century. Musical pieces from each era share certain characteristics and styles that are representative of the era. With a history of almost 300 years, symphonies today are viewed as the very pinnacle of classical music where Beethoven, Brahms, Mozart and other renowned composers were able to find a venue for transcending their creativities and overall influencing them heavily on their music. During the course of the 18th century, the tradition was  to write four-movement symphonies (Hepokoski \& Darcy, 2006). 

Throughout time, different styles have developed, each having features unique to themselves. Tilden (2013) notes the historical influence of composers with each other and how similar the methods of composing classical music are with pop music.  As a result, symphonies written in the early 20th century may be influenced by the great composers and compositions of the previous eras. Analyzing these musical relationships and comparing one to another is a research area that could be done through both manual and machine learning methods.

McFee, Barrington \& Lanckriet (2012) compare the use of context-based manual semantic annotation versus their proposed optimized content-based similarity learning framework. With machine learning, the use of high-quality training data without active user participation and the analysis of more data is possible than with feedback or survey data from active user participation. Human error in the analysis process can also be minimized with machine learning since human-supervised training is minimal. Corrêa \& Rodrigues (2016) shows the analysis of music features using machine learning techniques. According to the MIR community (Silla \& Freitas, 2009), the two main representation of music feature content are either audio-recorded or symbolic-based. The former employs the explicit recording of audio files while the latter uses symbolic data files such as MIDI or KERN.

SOMphony, a research paper by Azcarraga \& Flores (2016), aims to understand the relationship of symphonies between the same composer to denote style as well as to determine the similarities between symphonies of different periods of music to denote influences between time periods. Their research showed the relationships and influences between composers from 5 major musical periods, namely the Baroque Period, Classical Period, 19th Century music, Romantic Period and the 20th century. Their research focused on self-organizing maps (SOM) that are trained using 1-second music segments extracted from the 45 different symphonies. The trained SOM is then further processed by doing a k-means clustering of the node vectors, allowing quantitative comparison of music trajectories between symphonies. 

They used frequency counting in their research work for evaluation of each individual symphonies. Each time a 1 second music segment has a BMU inside a cluster, the frequency count for that cluster is incremented. In this way, only the clusters that are frequently visited or symphony will have a high frequency count. The frequency counts are then normalized by dividing the counts of a certain symphony by its total number of music segments. Once these normalized frequency counts are summarized, the resulting percentages can then be used to perform pairwise comparisons between symphonies.

Their research showed that using SOM is indeed helpful in visualizing the musical features of a symphony, making it easier to create insights about the relationships within the different pieces and composers. Their research concluded that a larger dataset would be needed to confirm whether the approach is indeed valid. 

SOMphony, however, did not take into consideration the notion of time. In time dimension, each instance represents a different time step and the attributes give values associated with that time (Witten  \& Frank, 2005).  To be able to generate time sensitive musical analysis, this research will add in the time dimension variable to the SOM and a new visualization in 3D space would need to be created.

In a research work by Maaten \& Hinton (2008), they introduce a new visualization technique for assigning data points in a two dimensional or three dimensional map called  t-Distributed Stochastic Neighbor Embedding or t-SNE, which is a variant of Stochastic Neighbor Embedding or SNE. Since t-SNE produces almost very distinct visualizations based on the experiment in their research work, this technique for visualizing individual symphonies may be more optimal than SOM in terms of both accuracy and efficiency.



\section{Research Objectives}
\label{sec:researchobjectives}
\subsection{General Objective}
\label{sec:generalobjective}
To visualize symphonies using time dimension
\subsection{Specific Objectives}
\label{sec:specificobjectives}
The research aims to:
\begin{enumerate}
\item Perform feature selection to decrease number of features for faster training time in machine learning;
\item Find time-dependent distance measures to compare trajectories;
\item Create a 3D visualization model for the music data;
\item Determine feasibility of t-SNE as another form of visualization for comparison of symphonies;
\end{enumerate}
\section{Scope and Limitations of the Research}
\label{sec:scopelimitations}
To be able to generate a self-organizing map, the proponents will use jAudio to extract 436 audio features from musical segments generated from the symphony (See Appendix C). The 436 audio features would be trimmed down through feature selection. By selecting only the n most influential features, the time it would take to train the SOM would not be as time consuming compared to using all 436 features, however, this may be at the cost of some of its accuracy in plotting the symphony’s musical trajectory. In extracting the features, the musical piece is divided into 1 second music segments in order to be uniform all throughout the piece and to avoid incomplete notes. A 0.5 second overlap is used to be able to consider transitions between each second. 

Different metrics for comparison will be used to evaluate which symphonies are similar. Together with the visual maps produced by either SOM or t-SNE, the proponents will then evaluate which metric shows results that reinforce the results seen from looking at the visual map. By doing this, the metrics that show more accurate results can be determined.

To create a 3D model to represent the symphony, the proponents will assign each generated SOM to a point in time and will be used to create a graph representing each map in a series. As a result of using the time dimension, this research will be able to better differentiate symphonies that use similar themes but at different periods of time in the composition.

Similar to SOMphony, the proponents will focus on representation of symphonies using SOMs for the purpose of comparison to other symphonies. T-SNE will also be used as a visualization technique for the musical features of each symphony. The proponents will then decide which among the two to be used for metric evaluations depending on the results.

\section{Significance of the Research}
\label{sec:significance}
As the study focuses on qualitative comparison and analysis of different symphonies, the results of the study will prove that music is indeed quantifiable as opposed to being solely qualitative in nature. 

The results of this study will also prove that the simplification of hours-long symphonies into a single visual representation in 3D space is possible.  It can prove that visualization can be achieved, allowing comparison of data along the time dimension. The results of the study may also help prove the benefits and possibilities of SOM when transitioning from 2D to 3D using time. The results may also verify if t-SNE is a good visualization technique for comparison of musical data.

Some possible future application of the results of this study would include the improvement of existing music information retrieval (MIR) techniques used by music databases. Similarly, this research can also be used to further improve the algorithms used by playlist managers for the retrieval of similar songs from music databases using the comparison of the trained SOMs.

\section{Research Methodology}
\label{sec:researchmethod}
This section contains phases and activities that will be performed to accomplish the research. The phases listed here will be arranged sequentially unless otherwise stated.

\subsection{Concept Formulation and Review of Related Literature}
This phase will concern the consolidation of the thesis requirements such as the objective of the research, the research problem to be tackled, and the scopes and limitations of such research. Research related to music comparison, machine learning algorithms in music and music visualization will be part of the Review of Related Literature.

\subsection{Data Gathering}
This phase will concern the gathering of the additional symphonies to be used for the research. The original music dataset for SOMphony is composed of 75 symphonies spread across 5 periods each having 3 composers. To expand the dataset, 2 symphonies will be added to each composer, summing up to a total of 5 symphonies per composer and 2 composers will also be added for each era summing up to a total of 125 symphonies. The proponents have decided to maintain 5 symphonies per composer so that the data set will an equal number of symphonies per composer. The process of selecting which symphonies to be added would be by random to have a better grasp of the general style of the composer. The audio files would be retrieved from online sources and physical means. The researchers would not take into consideration the file type and bitrate of the audio files since music data that is free for use is limited.

\subsection{Data Preparation}
To prepare the data, the symphony audio files are converted into wav files in preparation for splitting. WaveSplitter will be used in splitting the audio file into 1 second segments at intervals of 0.5 second. These segments would undergo feature extraction using jAudio. The result would be an xml file containing all the features determined for each segment. The researchers would then run a RegEx script to remove the unnecessary text and format it into comma-separated values (CSV) file in preparation for labeling. Since the proponents would conduct supervised learning, the data needs to be labelled. The labelling scheme is as follows: period, composer, symphony and audio number (P1C1S1-0001). 

The resulting CSV file was further cleaned by removing features that have the feature value 0 for all samples. Features that have at least one NaN value among its data was also removed. These features were removed since they were deemed to not be representative of the data. This leaves 276 features, from the original 534, to be used for training. Once completed, the the entire dataset is then normalized and clustered through Self-Organizing Maps.

\subsection{Feature Selection}
In this phase, the proponents will trim down the 436 features that jAudio has extracted. With decision trees, the top n nodes will be selected as the top n features. Principal component analysis (PCA), on the other hand, can be used to further reduce the number of features by merging columns that have similar data, until the dataset is only represented by a smaller number of features while still retaining the essence of the original data.. Aside from decision trees, the proponents will explore other statistical techniques such as PCA or Pearson correlation for feature selection which may prove more efficient or optimal than decision trees. By doing feature selection, the data set would have a uniform number of features for all symphonies and it would also enhance the efficiency of training the SOM.

\subsection{Machine Learning and Similarity Metrics}
This phase will concern training the SOM and t-SNE using the data with feature-selected features to reduce the dimensionality. The resulting maps can already be used to see similarities between composers, symphonies, or even musical periods. Different similarity metrics will be experimented and the results will be compared to the visual maps to see which metrics produce consistent results and to also be able to validate the similarities between the symphonies.

\subsection{Visualization}
The proponents will assign each BMU to a point on the generated SOM and this will be used to create a graph representing each map in a time series. As a result of using time dimension,  the proponents will be able to better differentiate symphonies that use similar themes but at different periods of time in the composition. T-SNE will also be used as a secondary visualization technique for visualizing the symphonies to see if this technique will also provide good visualization for symphonies like with SOM.

\subsection{Documentation}
This phase will be done all throughout the whole research timeframe. The previously mentioned stages and their corresponding findings would also be documented duly.

\section{Calendar of Activities}
Table 1.1 shows the time table for the activities involved with the research for 2017 and Table 1.2 shows the activities for 2018. The numbers represent the number of weeks worth of activity. The \# symbol represents the number of weeks allotted for the month.

\begin{center}
\begin{tabular}{ |p{3.7cm}|p{1.2cm}|p{1.2cm}|p{1.2cm}|p{1.2cm}|p{1.2cm}|p{1.2cm}|p{1.2cm}| }
 \hline
 \multicolumn{8}{|c|}{Calendar of Activities} \\
 \hline
 Activities for 2017& Jun&Jul&Aug&Sep&Oct&Nov&Dec\\
 \hline
Concept Formulation and RRL&\#\#\#& \#\#\#\#&\#&&&&
 \\
\hline
Data Gathering&&&\#&\#\#\#&\#\#&&
 \\
\hline
Data Preparation&&&&\#\#&\#\#&\#\#\#\#&\#\#
 \\
 \hline
Feature Selecion&&&&&&&
 \\
 \hline
Machine Learning and Similarity Metrics&&&&\#\#&\#\#&\#\#&\#
 \\
 \hline
Visualization&&&&&&&
 \\
 \hline
Documentation&\#\#&\#\#\#& \#\#&\#\#\#&\#\#\#\#&\#\#\#\#&\#\#
 \\
 \hline
\end{tabular}
\end{center}
Table 1.1 Timetable of Activities for 2017

\begin{center}
\begin{tabular}{ |p{3.7cm}|p{1.2cm}|p{1.2cm}|p{1.2cm}|p{1.2cm}|p{1.2cm}|p{1.2cm}|p{1.2cm}| }
 \hline
 \multicolumn{8}{|c|}{Calendar of Activities} \\
 \hline
 Activities for 2018& Jan&Feb&Mar&Apr&May&Jun&Jul\\
 \hline
Concept Formulation and RRL&&&&&&&
 \\
\hline
Data Gathering&&&&&&&
 \\
\hline
Data Preparation&&&&&&&    
 \\
 \hline
Feature Selecion&\#\#\#&\#\#\#\#&&&&&    
 \\
 \hline
Machine Learning and Similarity Metrics&&&\#\#\#\#&\#\#\#\#& \#\#\#\#&& 
 \\
 \hline
Visualization&&&&&\#\#\#\#&\#\#\#\#&
 \\
 \hline
Documentation&\#\#\#&\#\#\#\#&\#\#\#\#&\#\#\#\#&\#\#\#\#&\#\#\#\#&\#\#
 \\
 \hline
\end{tabular}
\end{center}
Table 1.2 Timetable of Activities for 2018